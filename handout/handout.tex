%TEMPLATE
%for Spring School 2025
%Don't change any macros given.
%The first place for your change is below line 250.
%See you at Spring School
%%%%%%%%%%%%%%%%%%%%%%%%%%%%%%%%%%%%%%%%%%%%%%%%%%%%%%%%%%%%%%%
%DO NOTÂ CHANGEÂ THIS
%
\documentclass[12pt]{report}
%%%%%%%%%%%%%%%%%%%%%%%%%%%%%%%%%%%%%%%%%%%
%PACKAGES
%basic
\usepackage[utf8x]{inputenc}
\usepackage[T1]{fontenc}
\usepackage{lmodern}
\usepackage[english]{babel}
\usepackage[unicode]{hyperref}%%sazba url
%apperance
%\usepackage{fullpage}
\usepackage[margin=0.75in]{geometry}
\usepackage[hyphenbreaks]{breakurl}
\usepackage{multicol}  %% for multiple columns in text
%schedule
\usepackage[usenames,dvipsnames]{color}
%math and usefull packages
\usepackage{amsfonts}
\usepackage{amsmath, amssymb}
\usepackage{graphicx}
\usepackage[linesnumbered, vlined]{algorithm2e}%algorithms
% tikz
\usepackage{tikz}


%%%% by participants:
%%%%%%%%%%%%%%%%%%%%%%%%%%%%%%%%%%%%%%%%%%%%%%%%%%%%%
%MACROS
%%%%%%%%%%%%%%%%%%%%%%%%%%%%%%%%%%%%%%%%%%%%%%%%%%%
%TITLE
\makeatletter

\def\pen{\penalty10000}


\newdimen\HeaderBoxMargin
\HeaderBoxMargin=3pt

\def\HeaderBox#1{
  \hbox{%
    \vrule width 0.4pt \hglue -0.4pt%
    \vbox{%
      \hrule height 0.4pt\pen\vskip0.5mm\hrule height 0.4pt\pen\smallskip%
      \begin{center}%
        % velmi velmi ošklivý hack, aby se text nedotýkal stran
        \leftskip=\the\HeaderBoxMargin plus 1fil%
        \rightskip=\the\HeaderBoxMargin plus 1fil%
        #1%
      \end{center}%
      \medskip\hrule height 0.4pt%
    }%
    \hglue -0.4pt \vrule width 0.4pt%
  }%
  \pen\smallskip%
}


\def\chapter*#1{\Heading{#1}}%


\long\def\literaturestart#1\literatureend{{
  \def\chapter*##1{\section{##1}\vspace*{-\parskip}}
  \begin{thebibliography}{9}
  \footnotesize\normalfont
    #1
  \end{thebibliography}
}}


% sázení section v obsahu vlevo a ne odsazeně
\def\l@section{\@dottedtocline{1}{0em}{2.3em}}

% okouše mezery z konce argumentu (protože unskip nefunguje v pdf obsahu)
\def\Trim@Impl#1#2 \end#3\Trim@End{\ifx\end#3\end#1\else\Trim{#2}\fi}
\def\Trim#1{\Trim@Impl{#1}#1\end{} \end\Trim@End}


\newcommand{\Heading}[1]{
\begin{center}
\bf \LARGE #1
\end{center}
\bigskip
\bigskip

}

%%%%%%%%%%%%%%%%%%%%%%%%%%%%%%%%%%%%%%%%%%%%%%%%%%%
%TITLE type

% jen obalí argumenty \Trim{} a pokud je #1 prázdné
% tak ho nastavĂ­ na \Trim{#2}\Trim{#3}
\long\def\Id#1\FirstName#2\Surname#3\Email#4\Serie#5\Title#6%
         \PaperAuthor#7\Link#8\Text#9\End{{%
  \IdImpl\ifx\XXX#1\XXX\Trim{#2}\Trim{#3}\else\Trim{#1}\fi
    \FirstName\Trim{#2}\Surname\Trim{#3}\Email\Trim{#4}%
    \Serie\Trim{#5}\Title\Trim{#6}%
    \PaperAuthor\Trim{#7}\Link\Trim{#8}\Text #9\End%
  }}

\def\Id@title#1{\medskip\\{\Large #1}}

\long\def\IdImpl#1\FirstName#2\Surname#3\Email#4\Serie#5\Title#6%
         \PaperAuthor#7\Link#8\Text#9\End{{%
  \def\nop{\Trim{}}%
  \def\testemail{#4}%
  \def\testserie{#5}%
  \def\testpaper{#7}%
  \def\testlink{#8}%
  %
  \phantomsection%
  \HeaderBox{%
    {\Large #2 #3}\pen%
    \ifx\testemail\nop\else%
      \smallskip\\{\tt #4}\pen%
    \fi%
    \ifx\testpaper\nop\else%
      \smallskip\\{Presented paper by #7}\pen%
    \fi%
    \ifx\testserie\nop%
      \Id@title{#6}%
      \addcontentsline{toc}{section}{#6 (\nobreak{#2} \nobreak{#3})}%
    \else%
      \Id@title{#6\penalty-2{} {\it as part of series} #5}%
      \addcontentsline{toc}{section}{#5: #6 (\nobreak{#2} \nobreak{#3})}%
    \fi%
    \ifx\testlink\nop\else%
      \smallskip\\{(\texttt{#8})}\pen%
    \fi%
    \label{#1}%
  }%
  \setcounter{theorem}{0}%
  {#9}%
}}

%name surname email {Title}
\newcommand{\Talk}[4]{%
  \Id\FirstName #1 \Surname #2 \Email #3
  \Serie
  \Title #4
  \PaperAuthor
  \Link
  \Text\End
}
%name surname email {author of the paper}{Title}{link}
\newcommand{\PaperTalk}[6]{%
  \Id\FirstName #1 \Surname #2 \Email #3
  \Serie
  \Title #5
  \PaperAuthor #4
  \Link #6
  \Text\End
}
%name surname email {author of the paper}{Title}{Serie}{link}
\newcommand{\SeriesTalk}[7]{%
  \Id\FirstName #1 \Surname #2 \Email #3
  \Serie #6
  \Title #5
  \PaperAuthor #4
  \Link #7
  \Text\End
}
%name surname email {Title}{Serie}
\newcommand{\SeriesOther}[5]{%
  \Id\FirstName #1 \Surname #2 \Email #3
  \Serie #5
  \Title #4
  \PaperAuthor
  \Link
  \Text\End
}
%%%%%%%%%%%%%%%%%%%%%%%%%%%%%%%%%%%%%%%%%%%%%%%%%%%
%Schedule macros
%Schaduled Talk: Talk Speaker_Name Speaker_Surname Time Duration 
\def\ST#1#2#3#4#5{\Event{#5}{#1}{#2 #3 \hskip 0pt plus 0.5fil \mbox{#4} \hskip 0pt plus 0.5fil \pageref{#2#3}}}
%%%%%%%%%%%%%%%%%%%%%%%%%%%%%%%%%%%%%%%%%%%%%%%%%%%
% Math enviroment definition.
\newcounter{theorem}
\newtheorem{definition}[theorem]{Definition}
\newtheorem{theorem}{Theorem}
\newtheorem{lemma}[theorem]{Lemma}
\newtheorem{observation}[theorem]{Observation}
\newtheorem{corollary}[theorem]{Corollary}
\newtheorem{conjecture}[theorem]{Conjecture}
\newenvironment{proof}{\par\textbf{Proof}}{ \rule{1pt}{0pt}   \hfill $\square$ \\}
\newtheorem{proposition}[theorem]{Proposition}
\newtheorem{note}[theorem]{Note}

\newtheorem{fact}[theorem]{Fact}
\newtheorem{exercise}[theorem]{Exercise}

%%%%%%%%%%%%%%%%%%%%%%%%%%%%%%%%%%%%%%%%%%%%%%%%%%%
% Dela sekce v jednotlivych kouskach sbornicku
\def\subsection#1{\par Not good idea. \smallskip}
\def\section#1{\vskip 9pt plus 2pt minus 1pt\centerline{\bf #1}\par\ignorespaces}
\def\paragraph#1{\par{\bf #1}\hskip 0.5em\ignorespaces}
%%%%%%%%%%%%%%%%%%%%%%%%%%%%%%%%%%%%%%%%%%%%%%%%%%%
%% Pro formatovani odstavcu
\setlength{\parindent}{0pt}
\setlength{\parskip}{5pt plus 1pt minus 1pt} % smallskip  size (was 0.5em)
\setlength{\parsep}{\parskip}
\def\sirka{0.89\textwidth}

\let\trivlist@old\trivlist
%\def\trivlist{\parskip=0pt\partopsep=0pt\trivlist@old}

\let\normalsize@old\normalsize
\def\normalsize{%
  \normalsize@old%
  \setlength{\abovedisplayskip}{5pt plus 3pt minus 2pt}% def 10 +2 -5 pt
  \setlength{\belowdisplayskip}{5pt plus 3pt minus 2pt}% def 10 +2 -5 pt
  \setlength{\abovedisplayshortskip}{0pt plus 3pt}% def 0 +3 pt
  \setlength{\belowdisplayshortskip}{0pt plus 3pt}% def 6 +3 -3 pt
}

\setlength{\itemsep}{0pt} % def 4.0pt plus 2.0pt minus 1.0pt
\setlength{\topsep}{0pt}
\setlength{\partopsep}{0pt}

%% Makro pro vkladani prispevku
\newcommand{\prispevek}[1]{
	\input{prispevky/#1}\vskip 3em
}
%%%%%%%%%%%%%%%%%%%%%%%%%%%%%%%%%%%%%%%%%%%%%%%%%%%
%% Math macros
\newcommand{\T}[1]{#1^\top\!} 
\newcommand{\calO}{{\mathcal O}}

\newcommand{\NP}{\ensuremath{\mathrm{NP}}}
\newcommand{\FPT}{\ensuremath{\mathrm{FPT}}}
\newcommand{\Pp}{\ensuremath{\mathrm{P}}}

\def\epsilon{\varepsilon}
\def\phi{\varphi}

\pdfstringdefDisableCommands{%
  \def\chi{\textchi}%
  \def\Delta{\textDelta}%
  \def\omega{\textomega}%
}

\makeatother

%%%%%%%%%%%%%%%%%%%%%%%%%%%%%%%%%%%%%%%%%%%%%%%%%%%%%%%%%%%%%%%%%%%%
%ENDÂ OF DOÂ NOTÂ CHANGEÂ THIS
%%%%%%%%%%%%%%%%%%%%%%%%%%%%%%%%%%%%%%%%%%%%%%%%%%%%%
%
%
%
%%%%%%%%%%%%%%%%%%%%%%%%%%%%%%%%%%%%%%%%%%%%%%%%%%%%%%%%%%%%%%%
%Your macros
%Please use as few as possible.
%\newcommand{\mbb}[1]{\ensuremath{\mathbb{#1}}}
\newcommand{\floor}[1]{\left\lfloor #1 \right\rfloor}
\newcommand{\floorphi}{\ensuremath{\floor{\phi}\hspace{-0.5ex}}}
%
%
%
%
%%%%%%%%%%%%%%%%%%%%%%%%%%%%%%%%%%%%%%%%%%%%%%%%%%%%%%%%%%%%%%%
%Your packages
%Please use as few as possible.
% \usepackage{yourfavouritepackage}
%
%
%
%
%%%%%%%%%%%%%%%%%%%%%%%%%%%%%%%%%%%%%%%%%%%%%%%%%%%%%
%DOCUMENT
\begin{document}
%%%%%%%%%%%%%%%%%%%%%%%%%%%%%%%%%%%%%%%%%%%%%%%%%%%%%%%%%%%%%%%
%Please fill the informations
% Choose only one of the following macros and comment out the rest.
% If you will present someones paper
%\PaperTalk{Your Name}{Surname}{your@email.cz}{Author of the article}{The Title}{http://link.springer.com/article/10.1007\%2FBF02579200}
%If you present a paper as a part of series.
%\SeriesTalk{Your Name}{Surname}{your@email.cz}{Author of the article}{The Title}{The Best Series}{http://link.springer.com/article/10.1007\%2FBF02579200}
%Other talk as a part of series.
%\SeriesOther{Your Name}{Surname}{your@email.cz}{The Title}{The Best Series}
%Some other talk
\Talk{Richard}{Mužík}{richard@imuzik.cz}{Integer cooperative game theory}
%%%%%%%%%%%%%%%%%%%%%%%%%%%%%%%%%%%%%%%%%%%%%%%%%%%%%%%%%%%%%%%
%The beginning of your write up.
\section{Introduction}
In this talk, we explore cooperative game theory in the context of integer-valued utilities and outcomes.
Motivated by the realism of indivisible resources, we examine how integer constraints impact both game properties and solution concepts.
We introduce and study two new classes of games—c-bounded and c-tight—and analyze their connections to established concepts like convexity and superadditivity.
Additionally, we propose several new integer-valued analogues of the Shapley value and compare their properties with the classical version.
The talk also includes an investigation of the nucleolus in the integer setting, highlighting key differences from the traditional, real-valued approach.

\begin{definition}
    The \emph{integer cooperative game} is a pair $(N, v_I)$, where $N$ is a finite set of players and $v_I: 2^N \to \mathbb{Z}$ is a characteristic function.
    Furthermore, $v_I(\emptyset) = 0$.
\end{definition}

\begin{definition}
    For integer cooperative game $G_I \in \mathcal{G}_I^n$, the following sets are defined:
    \begin{itemize}
        \item \emph{Integer imputation set}: $\mathcal{I}_{\mathbb{Z}}(G_I) = \mathcal{I}(G_I) \cap \mathbb{Z}^{n}$,
        \item \emph{Integer dual imputation set}: $\mathcal{I}_{\mathbb{Z}}^{\star}(G_I) = \mathcal{I}^{\star}(G_I) \cap \mathbb{Z}^{n}$.
    \end{itemize}
\end{definition}

\begin{definition}
    \label{def:c-tight}
    Let $c \in \mathbb{N}_0$.
    A cooperative game $(N,v)$ is \emph{$c$-tight} if it holds that
    \begin{displaymath}\label{eq:c-tight}
        \forall S \subseteq N: 0 \leq v(S) \leq c \land v(N) = c.
    \end{displaymath}
\end{definition}

\begin{definition}
    \label{def:c-bounded}
    Let $c \in \mathbb{N}_0$.
    A cooperative game $(N,v)$ is \emph{$c$-bounded} if it holds that
    \begin{displaymath}\label{eq:c-bounded}
        \forall S \subseteq N: 0 \leq v(S) \leq c.
    \end{displaymath}
\end{definition}

\begin{theorem}
    \label{theorem:number_of_k_games_c-tight}
    Let $c, n, k \in \mathbb{N}$ and $k \leq n$. The number of $c$-tight integer $k$-games is given by
    \begin{displaymath}
        \binom{\binom{n}{k}-1+c}{c}.
    \end{displaymath}
\end{theorem}

\begin{definition}
    \label{def:floor_shapley_value}
    For an integer cooperative game $G_I \in \mathcal{G}_I^n$, the \emph{Floor Shapley value} $\floorphi(G_I)$ is given by

    \vspace{-1cm}
    \begin{displaymath}
        \floorphi(G_I) = \floor{\phi(G_I)}.
    \end{displaymath}
\end{definition}


\begin{definition}
    \label{def:efficient_floor_shapley_value}
    For an integer cooperative game $G_I=(N,v_I) \in \mathcal{G}_I^n$, the \emph{Efficient Floor Shapley value} $\phi^E(G_I)$ is defined as follows:
    \begin{enumerate}
        \item Compute the Floor Shapley value $\floorphi(G_I)$ and the Shapley value $\phi(G_I)$.
        \item Compute the weights $w_i = \phi_i(G_I) - \floorphi_i(G_I)$ for all $i \in N$.
        \item Sort the weights in descending order such that if multiple players have the same weight, then their ordering is uniformly random.
        \item Each player receives his Floor Shapley value. Additionally, the top $k$ players, where $k = v_I(N) - \sum_{i \in N} \floorphi_i(v_I) = w(N)$, receive one extra unit.
    \end{enumerate}
\end{definition}


\begin{definition}
    \label{def:probabilistic_efficient_floor_shapley_value}
    For an integer cooperative game $G_I \in \mathcal{G}_I^n$, the \emph{Probabilistic Efficient Floor Shapley value} $\phi^{\mathbb{E}}(v)$ is defined as follows:
    \begin{enumerate}
        \item Compute the Floor Shapley value $\floorphi(G_I)$ and the Shapley value $\phi(G_I)$.
        \item Compute the remainders $\tilde{p}_i = \phi_i(G_I) - \floorphi_i(G_I)$ for all $i \in N$.
        \item Compute the probabilities $p_i = \frac{\tilde{p}_i}{\sum_{j \in N} \tilde{p}_j}$ for all $i \in N$.
        \item Each player receives his Floor Shapley value and additionally, each unit of the remainder with probability $p_i$, i.e., each unit of $\tilde{p}(N)=\sum_{j \in N} \tilde{p}_j$ is given to player $i$ with probability $p_i$.
    \end{enumerate}
\end{definition}


\begin{theorem}
    \label{theorem:probabilistic_efficient_floor_shapley_properties}
    The Probabilistic Efficient Floor Shapley value $\phi^{\mathbb{E}}$  satisfies the following properties for all integer games $(N,v_I),(N,w_I) \in \mathcal{G}_I^n$:
    \begin{enumerate}
        \item The expected value is the same as the Shapley value:
            \begin{displaymath}
                \label{eq:probabilistic_efficient_floor_shapley_properties:expected_value}
                \mathbb{E}[\phi^{\mathbb{E}}(v_I)] = \phi(v_I),
            \end{displaymath}
        \item Axiom of efficiency:
            \begin{displaymath}
                \sum_{i \in N}\phi_{i}^{\mathbb{E}}(v_I) = v_I(N),
            \end{displaymath}
        \item Axiom of expected symmetry:
            \begin{displaymath}
                \forall i,j \in N (\forall S \subseteq N \setminus \{i,j\}: v_I(S \cup i) = v_I(S \cup j)) \Rightarrow \mathbb{E}[\phi_{i}^{\mathbb{E}}(v_I)] = \mathbb{E}[\phi_{j}^{\mathbb{E}}(v_I)],
            \end{displaymath}
        \item Axiom of null player:
            \begin{displaymath}
                \forall i \in N(\forall S \subseteq N: v_I(S)=v_I(S \cup i)) \implies \phi_{i}^{\mathbb{E}}(v_I) = 0,
            \end{displaymath}
        \item Axiom of expected additivity:
            \begin{displaymath}
                \mathbb{E}[\phi^{\mathbb{E}}(v_I+w_I)] = \mathbb{E}[\phi^{\mathbb{E}}(v_I)] + \mathbb{E}[\phi^{\mathbb{E}}(w_I)].
            \end{displaymath}
    \end{enumerate}
\end{theorem}

\begin{definition}
    \label{def:CLS}
    Let $||\bullet||$ be a vector norm. The \emph{Closest Lattice Shapley (CLS)} value of integer cooperative game $G_I \in \mathcal{G}_I^n$ is given by
        $\phi^\mathcal{W}(G_I)  = \min_{x \in \mathcal{W}_\mathbb{Z}(G_I)} ||\phi(G_I) - x||$.

\end{definition}

\begin{proposition}
    The CLS value:
    \begin{itemize}
        \item exists for all integer games $G_I \in \mathcal{G}_I^n$,
        \item is not unique in general,
        \item depends on the choice of the norm,
        \item is different from the Efficient Floor Shapley value.
    \end{itemize} 
\end{proposition}


\begin{definition}
    \label{def:integer_nucleolus}
    For an integer cooperative game $G_I \in \mathcal{G}_I^n$, the \emph{integer nucleolus} $\eta_\mathbb{Z}(G_I)$ is defined as
    \begin{displaymath}
        \eta_{\mathbb{Z}}(G_I) = \left\{ x \in \mathcal{I}_\mathbb{Z}(G_I) \mid \forall y \in \mathcal{I}_\mathbb{Z}(G_I): \Theta_\mathbb{Z}(x) \preceq_{lex} \Theta_\mathbb{Z}(y)\right\}.
    \end{displaymath}
\end{definition}

\begin{theorem}
    \label{thm:integer_nucleolus_existence}
    For an integer cooperative game $G_I \in \mathcal{G}_I^n$, it holds

    \begin{displaymath}
        \mathcal{I}_{\mathbb{Z}}(G_I) \neq \emptyset \implies \eta_\mathbb{Z}(G_I) \neq \emptyset.
    \end{displaymath}
\end{theorem}

\begin{theorem}\label{thm:integer_nucleolus_nonemptiness}
    For an integer cooperative game $G_I \in \mathcal{G}_I^n$, it holds
    \begin{displaymath}
        \eta_\mathbb{Z}(G_I) \neq \emptyset \iff \eta(G_I) \neq \emptyset.
    \end{displaymath}
\end{theorem}

\begin{theorem}
    \label{thm:integer_nucleolus_not_unique}
    For an integer cooperative game $G_I \in \mathcal{G}_I^n$ the nucleolus is not necessarily a single point solution concept.
\end{theorem}


%%%%%%%%%%%%%%%%%%%%%%%%%%%%%%%%%%%%%%%%%%%%%%%%%%%%%%%%%%%%%%%
%In case you would like to have some refences.
%It is usually not necessary.
%Please, uncomment/comment start and end of the bibliography
%\literaturestart
	
%please add a literature in given format as in examples:
%\bibitem{latexcompanion} 
%Michel Goossens, Frank Mittelbach, and Alexander Samarin. 
%\textit{The \LaTeX\ Companion}. 
%Addison-Wesley, Reading, Massachusetts, 1993.
% 
% 
%\bibitem{knuthwebsite} 
%Knuth: Computers and Typesetting,
%\\\texttt{http://www-cs-faculty.stanford.edu/\~{}uno/abcde.html}

%\literatureend
%%%%%%%%%%%%%%%%%%%%%%%%%%%%%%%%%%%%%%%%%%%%%%%%%%%%%%%%%%%%%%%
%end of the document
%PLEASEÂ DO NOTÂ CHANGE
\end{document}
