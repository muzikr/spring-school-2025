%%%%%%%%%%%%%%%%%%%%%%%%%%%%%%%%%%%%%%%%%
% Focus Beamer Presentation
% LaTeX Template
% Version 1.0 (8/8/18)
%
% This template has been downloaded from:
% http://www.LaTeXTemplates.com
%
% Original author:
% Pasquale Africa (https://github.com/elauksap/focus-beamertheme) with modifications by
% Vel (vel@LaTeXTemplates.com)
%
% Template license:
% GNU GPL v3.0 License
%
% Important note:
% The bibliography/references need to be compiled with bibtex.
%
%%%%%%%%%%%%%%%%%%%%%%%%%%%%%%%%%%%%%%%%%

%----------------------------------------------------------------------------------------
%	PACKAGES AND OTHER DOCUMENT CONFIGURATIONS
%----------------------------------------------------------------------------------------
\documentclass{beamer}
%\documentclass[handout]{beamer}

\usepackage{mathtools}
\usepackage{comment}

\usetheme{focus} % Use the Focus theme supplied with the template
% Add option [numbering=none] to disable the footer progress bar
% Add option [numbering=fullbar] to show the footer progress bar as always full with a slide count

% Uncomment to enable the ice-blue theme
%\definecolor{main}{RGB}{92, 138, 168}
%\definecolor{background}{RGB}{240, 247, 255}

\definecolor{mygreen}{RGB}{0, 128, 0}
\definecolor{myblue}{RGB}{51, 51, 204}
\definecolor{myviolet}{RGB}{255, 0, 255}

\newcommand{\red}[1]{\textcolor{red}{#1}}
\newcommand{\green}[1]{\textcolor{mygreen}{#1}}
\newcommand{\blue}[1]{\textcolor{myblue}{#1}}
\newcommand{\violet}[1]{\textcolor{myviolet}{#1}}

\newcommand{\R}{\mathbb{R}}
\newcommand{\Z}{\mathbb{Z}}
\newcommand{\Rn}[1][n]{\mathbb{R}^{#1}}
\newcommand{\bx}{\textbf{x}}
\newcommand{\by}{\textbf{y}}
\newcommand{\bz}{\textbf{z}}
\newcommand{\bu}{\textbf{u}}
\newcommand{\bv}{\textbf{v}}
\newcommand{\I}{\mathcal{I}}

\newcommand{\floor}[1]{\left\lfloor #1 \right\rfloor}
\newcommand{\floorphi}{\ensuremath{\floor{\phi}\hspace{-0.5ex}}}

\DeclareMathOperator{\E}{\mathbb{E}\,}

%------------------------------------------------

\usepackage{booktabs} % Required for better table rules
%\usepackage{slashbox}
%\usepackage{ulem,centernot}


%----------------------------------------------------------------------------------------
%	 TITLE SLIDE
%----------------------------------------------------------------------------------------

\title{Integer cooperative game theory}

\author{Richard Mužík}

\titlegraphic{\hspace*{5cm}\includegraphics[scale=0.5]{../img/logo-en.pdf}} % Optional title page image, comment this line to remove it

\institute{richard@imuzik.cz}


%------------------------------------------------

\begin{document}

%------------------------------------------------

\begin{frame}
	\maketitle % Automatically created using the information in the commands above
\end{frame}

%----------------------------------------------------------------------------------------
%	 SECTION 1
%----------------------------------------------------------------------------------------

\section{Introduction} % Section title slide, unnumbered

%------------------------------------------------

\begin{frame}{Cooperative game theory}
    %\pause

    \begin{block}{Cooperative game}
        A \textbf{cooperative game} is an ordered pair $(N,v)$, where $N=\{1,\dots,n\}$ is a finite set of players and $v\colon 2^N \to \mathbb{R}$ is the characteristic function.
        Further, $v(\emptyset) = 0$.
    \end{block}

    \begin{itemize}
        \item $S \subseteq N$ ... coalition
        \item $v(S)$ ... values of coalition
    \end{itemize}

    %\pause

    Goal: \textit{To distribute the value of the game among players}

    %\pause

    \begin{itemize}
	    \item \textbf{Payoff vector} $\bx \in \Rn$
	    \begin{itemize}
	        \item $x_i$ represents payoff of player $i$
	    \end{itemize}
        %\pause
	    \item Vector $\bx \in \Rn$ is \textbf{efficient}, if $\sum_{i \in N}x_i = v(N)$
	    \begin{itemize}
	        \item Usually, we distribute $v(N)$
	    \end{itemize}
	    \item Vector $\bx \in \Rn$ is \textbf{individually rational}, if $x_i \geq v(i)$
	    \begin{itemize}
	        \item players prefer $x_i$ over $v(i)$
	    \end{itemize}
	\end{itemize}
\end{frame}

%------------------------------------------------

%------------------------------------------------

\begin{frame}{Cooperative game theory - solution concepts}
    %\pause

    \textit{But firstly...}

    %\pause

    \begin{block}{Imputation set}
        For cooperative game $(N,v)$, we define the following sets:
        \begin{itemize}
            \item \textbf{Preimputation}: $\I^*(v) = \left\{x \in \Rn \mid x(N) = v(N) \right\}$
            \item \textbf{Imputation}: $\I(v) = \left\{x \in \I^*(v) \mid \forall i \in N: x_i \geq v(i)\right\}$
        \end{itemize} 
    \end{block}

    %\pause

    \textit{Possible fair solution...}

    %\pause

    \begin{block}{The Shapley value}
        For a cooperative game $(N,v)$, the \textbf{Shapley value} $\phi(v)$ of player $i$ is
        \begin{displaymath}
            \phi_i(v) = \sum_{S \subseteq N \setminus i}\frac{s!(n-s-1)!}{n!}\left(v(S \cup i) - v(S)\right)    
        \end{displaymath}
        
    \end{block}
\end{frame}

%------------------------------------------------

\begin{frame}{Cooperative game theory - solution concepts}
    %\pause
    \textit{It is possible to define it using its properties...}
	%\pause
    \begin{block}{The Shapley value}
       The \textbf{Shapley value} $\phi(v)$ is the only function $f \colon \Gamma^n \to \mathbb{R}$ satisfying for all games $(N,v),(N,w)$:
        \begin{enumerate}
            \item (AXIOM OF EFFICIENCE)
            \begin{itemize}
                \item $\sum_{i \in N}f_i(v) = v(N)$
            \end{itemize}
            \item (AXIOM OF SYMMETRY)
            \begin{itemize}
                \item$\forall i,j \in N$ $(\forall S \subseteq N \setminus \{i,j\}: v(S \cup i) = v(S \cup j)) \implies f_i(v) = f_j(v)$
            \end{itemize}
            \item (AXIOM OF NULL PLAYER)
            \begin{itemize}
                \item $\forall i \in N$ $(\forall S \subseteq N: v(S) = v(S \cup i)) \implies f_i(v)=0$
            \end{itemize}
            \item (AXIOM OF ADDITIVITY)
            \begin{itemize}
                \item $v,w \in \Gamma^n: f(v+w)=f(v)+f(w)$
            \end{itemize}
        \end{enumerate}

    \end{block}
\end{frame}

%------------------------------------------------

%------------------------------------------------

\begin{frame}{Cooperative game theory - solution concepts}
    \begin{enumerate}
		\item We fix one construction of $N$
		\begin{itemize}
			\item $\sigma \in \Sigma_n$ ... represent it by permutation
			\item $\sigma(i)$ ... the order, in which player i enters the coalition
		\end{itemize}
		\item We compute player's payments
		\begin{itemize}
			\item $m^\sigma_v$ ... \textbf{marginal vector}
			\item $\left(m^\sigma_v\right)_i=v\left(S_{\sigma(i)}\cup i\right)-v\left(S_{\sigma(i)}\right)$
			\begin{itemize}
				\item $S_{\sigma(i)}= \left\{j \in N \mid \sigma(j) < \sigma(i)\right\}$ ... predecessors of $i$ under $\sigma$
			\end{itemize}
		\end{itemize}
		\item We consider all combinations
		\begin{itemize}
			\item $\mathcal{W}(v)=conv\left\{m^{\sigma}_{v}\mid \sigma \in \Sigma_n\right\}$ ... \textbf{Webber set}
		\end{itemize}
	\end{enumerate}
	\begin{itemize}
		\item It holds: $\phi(v)=\sum_{\sigma \in \Sigma_n} \frac{m^\sigma_v}{n!}$
	\end{itemize}
	\begin{block}{Relation between the Shapley value and the Webber set}
		For a cooperative game $(N,v)$ it holds:
		\[\phi(v) \in \mathcal{W}(v)\]
		Moreover, $\phi(v)$ is the center of gravity of $\mathcal{W}(v)$.
	\end{block}
\end{frame}

%------------------------------------------------

%------------------------------------------------

\begin{frame}{Cooperative game theory - solution concepts}
    %\pause

    \begin{block}{Excess}
            Let $(N,v)$ be an essential game, i.e., $v(N) \geq \sum_{i \in N}v(i)$, $x \in \mathcal{I}(v)$ and $\emptyset \neq S \subseteq N$.
            The \emph{excess $e(S,x,v)$ of coalition $S$ in game $(N,v)$ with respect to payoff vector $x$} is defined as $e(S,x,v) = v(S) - x(S)$.
    \end{block}

    %\pause

    \begin{itemize}
        \item We further need to define the \emph{vector of excesses} for every $x \in \mathbb{R}^n$ as $\Theta(x)=\left(e(S_1,x,v), \dots, e(S_{2^{n}-2},x,v)\right)$, where the excesses are ordered in decreasing order.
    \end{itemize}

    %\pause

    \begin{block}{Nucleolus}
        For a cooperative game $(N,v)$, the \emph{nucleolus} $\eta(v)$ is defined as
        \begin{displaymath}
            \eta(v) = \left\{ x \in \mathcal{I}(v) \mid \forall y \in \mathcal{I}(v): \Theta(x) \preceq_{lex} \Theta(y)\right\},
        \end{displaymath}
        where $\preceq_{lex}$ is the lexicographical ordering of vectors.
    \end{block}
\end{frame}

%------------------------------------------------

%------------------------------------------------

\begin{frame}{Cooperative game theory - integer setting}
    %\pause

    \textit{The goal of integer cooperative game theory is to study and find integer solution concepts...}

    %\pause

    \begin{block}{Integer cooperative game}
        The \textbf{integer cooperative game} is a pair $(N, v_I)$, where $N$ is a finite set of players and $v_I: 2^N \to \Z$ is a characteristic function.
        Furthermore, $v_I(\emptyset) = 0$.
    \end{block}

    %\pause

    \begin{itemize}
        \item Does not ensure the integrality of the solution concepts
        \item Introduce new solution concepts based on the already known ones, ensuring integrality
    \end{itemize}

\end{frame}

%------------------------------------------------

%------------------------------------------------

\begin{frame}{Cooperative game theory - integer setting}
    %\pause

    \textit{Multi point sets...}

    %\pause

    \begin{block}{Imputation set}
        For integer cooperative game $G_I \in \mathcal{G}_I^n$, the \emph{Integer imputation set} set is defined as $\mathcal{I}_{\Z}(G_I) = \mathcal{I}(G_I) \cap \Z^{n}$.
    \end{block}

    %\pause

    \begin{block}{Solution concepts}
        For integer cooperative game $G_I \in \mathcal{G}_I^n$, the following solution concepts are defined:
        \begin{itemize}
            \item \emph{Integer core}: $\mathcal{C}_{\Z}(G_I) = \mathcal{C}(G_I) \cap \Z^{n}$,
            \item \emph{Integer D-Core}: $DC_{\Z}(G_I) = \mathcal{I}_{\Z}(G_I) \setminus dom(\mathcal{I}_{\Z}(G_I))$,
            \item \emph{Integer Weber set}: $\mathcal{W}_{\Z}(G_I) = \mathcal{W}(G_I) \cap \Z^{n}$,
            \item \dots
        \end{itemize}
    \end{block}

    %\pause

\end{frame}

%------------------------------------------------


%----------------------------------------------------------------------------------------
%	 SECTION 3
%----------------------------------------------------------------------------------------

\section{Results}

%------------------------------------------------

\begin{frame}{Floor Shapley value}
    %\pause

    \begin{block}{Floor Shapley value}
        For an integer cooperative game $G_I \in \mathcal{G}_I^n$, the \textbf{Floor Shapley value} $\floorphi(G_I)$ is given by $\floorphi(G_I) = \floor{\phi(G_I)}$.
    \end{block}

    %\pause

    \begin{block}{Theorem $16$}
        The Floor Shapley value $\floorphi(v_I)$ satisfies the following properties for all integer games $(N,v_I),(N,w_I) \in \mathcal{G}_I^n$:
        \begin{enumerate}
            \item Axiom of near efficiency: $v_I(N) - n \leq \sum_{i \in N} \floorphi_i(v_I) \leq v_I(N)$,
            \item Axiom of symmetry: $\forall i,j \in N (\forall S \subseteq N \setminus \{i,j\}: v_I(S \cup i) = v_I(S \cup j)) \Rightarrow \floorphi_{i}(v_I) = \floorphi_{j}(v_I)$,
            \item Axiom of null player: $\forall i \in N(\forall S \subseteq N: v_I(S)=v_I(S \cup i)) \implies \floorphi_{i}(v_I) = 0$,
            \item Axiom of near additivity: $\floorphi(v_I+w_I) = \floor{\phi(v_I) + \phi(w_I)}$.
        \end{enumerate}
    \end{block}

\end{frame}

%------------------------------------------------

%------------------------------------------------

\begin{frame}{Efficient Floor Shapley value}
    %\pause
    \begin{itemize}
        \item The idea of preserving the efficiency...
        %\pause
        \item Redistribution of the remaining value...
    \end{itemize}

    %\pause

    \begin{block}{Efficient Floor Shapley value}
        For an integer cooperative game $G_I=(N,v_I) \in \mathcal{G}_I^n$, the \textbf{Efficient Floor Shapley value} $\phi^E(G_I)$ is defined as follows:
        \begin{enumerate}
            \item Compute the Floor Shapley value $\floorphi(G_I)$ and the Shapley value $\phi(G_I)$.
            \item Compute the weights $w_i = \phi_i(G_I) - \floorphi_i(G_I)$ for all $i \in N$.
            \item Sort the weights in descending order such that if multiple players have the same weight, then their ordering is uniformly random.
            \item Each player receives his Floor Shapley value. Additionally, the top $k$ players, where $k = v_I(N) - \sum_{i \in N} \floorphi_i(v_I) = w(N)$, receive one extra unit.
        \end{enumerate}
    \end{block}
\end{frame}

%------------------------------------------------

%------------------------------------------------

\begin{frame}{Efficient Floor Shapley value - properties}
    %\pause

    \begin{block}{Theorem $17$}
        The Efficient Floor Shapley value $\phi^E$ satisfies the following properties for all integer games $(N,v_I) \in \mathcal{G}_I^n$:
        \begin{enumerate}
            \item Axiom of efficiency: $\sum_{i \in N}\phi_{i}^E(v_I) = v_I(N)$,
            \item Axiom of expected symmetry: $\forall i,j \in N (\forall S \subseteq N \setminus \{i,j\}: v_I(S \cup i) = v_I(S \cup j)) \Rightarrow \E[\phi^E_{i}(v_I)] = \E[\phi^E_{j}(v_I)]$,
            \item Axiom of null player: $\forall i \in N(\forall S \subseteq N: v_I(S)=v_I(S \cup i)) \implies \phi_{i}^E(v_I) = 0$.
        \end{enumerate}
    \end{block}

    %\pause

    \begin{itemize}
        \item No axiom of additivity % TODO: v nekterych det a v sub random - example 
    \end{itemize}
\end{frame}

%------------------------------------------------

%------------------------------------------------

\begin{frame}{Probabilistic Efficient Floor Shapley value}
    \begin{itemize}
        \item Same idea, different approach...
    \end{itemize}

    %\pause

    \begin{block}{Probabilistic Efficient Floor Shapley value}
        For an integer cooperative game $G_I \in \mathcal{G}_I^n$, the \textbf{Probabilistic Efficient Floor Shapley value} $\phi^{\E}(v)$ is defined as follows:
        \begin{enumerate}
            \item Compute the Floor Shapley value $\floorphi(G_I)$ and the Shapley value $\phi(G_I)$.
            \item Compute the remainders $\tilde{p}_i = \phi_i(G_I) - \floorphi_i(G_I)$ for all $i \in N$.
            \item Compute the probabilities $p_i = \frac{\tilde{p}_i}{\sum_{j \in N} \tilde{p}_j}$ for all $i \in N$.
            \item Each player receives his Floor Shapley value and additionally, each unit of the remainder with probability $p_i$, i.e., each unit of $\tilde{p}(N)=\sum_{j \in N} \tilde{p}_j$ is given to player $i$ with probability $p_i$.
        \end{enumerate}
    \end{block}
\end{frame}

%------------------------------------------------

%------------------------------------------------

\begin{frame}{Probabilistic Efficient Floor Shapley value - properties}
    %\pause

    \begin{block}{Theorem $18$}
        The Probabilistic Efficient Floor Shapley value $\phi^{\E}$  satisfies the following properties for all integer games $(N,v_I),(N,w_I) \in \mathcal{G}_I^n$:
        \begin{enumerate}
            \item The expected value is the same as the Shapley value: $\E[\phi^{\E}(v_I)] = \phi(v_I)$,
            \item Axiom of efficiency: $\sum_{i \in N}\phi_{i}^{\E}(v_I) = v_I(N)$,
            \item Axiom of expected symmetry: $\forall i,j \in N (\forall S \subseteq N \setminus \{i,j\}: v_I(S \cup i) = v_I(S \cup j)) \Rightarrow \E[\phi_{i}^{\E}(v_I)] = \E[\phi_{j}^{\E}(v_I)]$,
            \item Axiom of null player: $\forall i \in N(\forall S \subseteq N: v_I(S)=v_I(S \cup i)) \implies \phi_{i}^{\E}(v_I) = 0$,
            \item Axiom of expected additivity: $\E[\phi^{\E}(v_I+w_I)] = \E[\phi^{\E}(v_I)] + \E[\phi^{\E}(w_I)]$.
        \end{enumerate}
    \end{block}

\end{frame}

%------------------------------------------------

%------------------------------------------------

\begin{frame}{Probabilistic Efficient Floor Shapley value - properties}
    %TODO: split the slide (first up to colors, second finish)
    \begin{block}{Theorem $18$}
        \begin{enumerate}
            \item The expected value is the same as the Shapley value: $\E[\phi^{\E}(v_I)] = \phi(v_I)$.
        \end{enumerate}
    \end{block}
    Proof: After all the rounds the player gets
    \begin{displaymath}
        \E[\phi_{i}^{\E}(v_I)] = \floorphi_i(v_I) + \blue{\tilde{p}(N)} * \red{p_i} = \floorphi_i(v_I) + \green{\tilde{p}_i} = \phi_i(v_I).
    \end{displaymath}
    \begin{itemize}
        \item $\blue{\tilde{p}(N) = \sum_{j \in N} \tilde{p}_j}$
        \item $\red{p_i = \frac{\tilde{p}_i}{\sum_{j \in N} \tilde{p}_j}}$
        \item $\green{\tilde{p}_i = \phi_i(G_I) - \floorphi_i(G_I)}$
    \end{itemize}
\end{frame}

%------------------------------------------------

%------------------------------------------------

\begin{frame}{Probabilistic Efficient Floor Shapley value - properties}
    \begin{block}{Theorem $18$}
        \begin{enumerate}
            \setcounter{enumi}{1}
            \item Axiom of efficiency: $\sum_{i \in N}\phi_{i}^{\E}(v_I) = v_I(N)$.
        \end{enumerate}
    \end{block}
    Proof: Directly from the definition as well as from the motivation.

    \begin{block}{Theorem $18$}
        \begin{enumerate}
            \setcounter{enumi}{3}
            \item Axiom of null player: $\forall i \in N(\forall S \subseteq N: v_I(S)=v_I(S \cup i)) \implies \phi_{i}^{\E}(v_I) = 0$.
        \end{enumerate}
    \end{block}
    Proof:
    \begin{itemize}
        \item Assume $i$ is null player.
        \item Floor Shapley value $\floorphi_i(v_I) = 0$.
        \item Additionally the probability that $i$ receive $1$ is $0$.
    \end{itemize}

\end{frame}

%------------------------------------------------

%------------------------------------------------

\begin{frame}{Probabilistic Efficient Floor Shapley value - properties}
    \begin{block}{Theorem $18$}
        \begin{enumerate}
            \setcounter{enumi}{2}
            \item Axiom of expected symmetry: $\forall i,j \in N (\forall S \subseteq N \setminus \{i,j\}: v_I(S \cup i) = v_I(S \cup j)) \Rightarrow \E[\phi_{i}^{\E}(v_I)] = \E[\phi_{j}^{\E}(v_I)]$.
        \end{enumerate}
    \end{block}

    Proof:

    \begin{itemize}
        \item Consider $i,j \in N: \phi_i(v_I) = \phi_j(v_I)$
        \item $\implies \floorphi_i(V_I) = \floorphi_j(v_I)$
        \item $\implies \tilde{p}_i = \tilde{p}_j$
        \item Combine with $1.$ (i.e. $\E[\phi^{\E}(v_I)] = \phi(v_I)$)
        \item We get
        \begin{displaymath}
            \E[\phi_{i}^{\E}(v_I)] = \phi_i(v_I) = \phi_j(v_I) = \E[\phi_{j}^{\E}(v_I)].
        \end{displaymath}
    \end{itemize}
\end{frame}

%------------------------------------------------

%------------------------------------------------

\begin{frame}{Probabilistic Efficient Floor Shapley value - properties}
    \begin{block}{Theorem $18$}
        \begin{enumerate}
            \setcounter{enumi}{4}
            \item Axiom of expected additivity: $\E[\phi^{\E}(v_I+w_I)] = \E[\phi^{\E}(v_I)] + \E[\phi^{\E}(w_I)]$.
        \end{enumerate}
    \end{block}

    Proof:

    \begin{itemize}

        \item Again use $1.$ (i.e. $\E[\phi^{\E}(v_I)] = \phi(v_I)$)
        \item We get
        \begin{displaymath}
            \E[\phi^{\E}(v_I+w_I)] = \phi(v_I+w_I) = \phi(v_I) + \phi(w_I) = \E[\phi^{\E}(v_I)] + \E[\phi^{\E}(w_I)].
        \end{displaymath}
    \end{itemize}

\end{frame}

%------------------------------------------------

%------------------------------------------------

\begin{frame}{Closest lattice Shapley}
    %\pause
    \begin{itemize}
        \item Different approach using integer programming
        %\pause
        \item Uses relation of the Shapley value and the Weber set
    \end{itemize}

    %\pause

    \begin{block}{Closest Lattice Shapley value}
        Let $||\bullet||$ be a vector norm. The \textbf{Closest Lattice Shapley (CLS)} value of integer cooperative game $G_I \in \mathcal{G}_I^n$ is given by

        \begin{displaymath}
            \phi^\mathcal{W}(G_I)  = \min_{x \in \mathcal{W}_\Z(G_I)} ||\phi(G_I) - x||.
        \end{displaymath}
    \end{block}
\end{frame}

%------------------------------------------------

%------------------------------------------------

\begin{frame}{Closest lattice Shapley - properties}
    %\pause

    \begin{itemize}
        \item Does it even exist?
    \end{itemize}

    %\pause

    \begin{block}{Claim $19$}
        The CLS value exists for all integer games $G_I \in \mathcal{G}_I^n$.
    \end{block}

    Proof:
    \begin{itemize}
        \item Real question: \textit{Is integer Weber always set non-empty?}
        \item $\left(m^\sigma_{v_I}\right)_i=v\left(S_{\sigma(i)}\cup i\right)-v\left(S_{\sigma(i)}\right) \in \Z$,
        \item $\mathcal{W}(v_I)=conv\left\{m^{\sigma}_{v_I}\mid \sigma \in \Sigma_n\right\}$,
        \item $\mathcal{W}_{\Z}(G_I) = \mathcal{W}(G_I) \cap \Z^{n}$,
        \item $\implies$ at least the points from which we do convex hull.
    \end{itemize}
\end{frame}

%------------------------------------------------

%------------------------------------------------

\begin{frame}{Closest lattice Shapley - properties}

    \begin{itemize}
        \item What about the properties?
    \end{itemize}

    %\pause

    \begin{block}{Theorem $20$}
        The CLS value $\phi^\mathcal{W}(v_I)$ satisfies the following properties for all integer games $(N,v_I) \in \mathcal{G}_I^n$:
        
        \begin{enumerate}
            \item Axiom of efficiency: $\sum_{i \in N}(\phi^W)_{i}(v_I) = v_I(N)$,
            \item Axiom of null player: $\forall i \in N(\forall S \subseteq N: v_I(S)=v_I(S \cup i)) \implies (\phi^W)_{i}(v_I) = 0$.
        \end{enumerate}
    \end{block}

    %\pause

    \begin{itemize}
        \item No symmetry and additivity
    \end{itemize}
\end{frame}

%------------------------------------------------

%------------------------------------------------

\begin{frame}{Closest lattice Shapley - properties}
    %\pause

    \begin{itemize}
        \item What other properties...
    \end{itemize}

    %\pause

    \begin{block}{Claims $21$ - $24$}
        For integer game $G_I \in \mathcal{G}_I^n$, the CLS value $\phi^\mathcal{W}(G_I)$:
        \begin{itemize}
            \item is not unique in general,
            \item is not always an extreme point of the Weber set $\mathcal{W}(G_I)$,
            \item depends on the choice of the norm.
        \end{itemize}
    \end{block}
\end{frame}

%------------------------------------------------

%------------------------------------------------

\begin{frame}{Closest lattice Shapley - properties}

    \begin{block}{$G_I^{\star}=(N,v_I)$}
        \begin{itemize}
            \item $N=\{1,2\}$,
            \item $v_I(\{1\})=v_I(\{2\})=1$, $v_I(N)=5$,
            \item $\phi(v_I)=(2.5,2.5)$.
        \end{itemize}
    \end{block}

    \centering
    \includegraphics[height=5cm]{../img/figure_weber_set_example.png}
\end{frame}

%------------------------------------------------

%------------------------------------------------

\begin{frame}{Closest lattice Shapley - properties}

    \begin{block}{Claims $21$ - $24$}
        For integer game $G_I \in \mathcal{G}_I^n$, the CLS value $\phi^\mathcal{W}(G_I)$:
        \begin{itemize}
            \item is not unique in general,
            \item is not always an extreme point of the Weber set $\mathcal{W}(G_I)$.
        \end{itemize}
    \end{block}

    \centering
    \includegraphics[height=5cm]{../img/figure_weber_set_example.png}
\end{frame}

%------------------------------------------------

%------------------------------------------------

\begin{frame}{Closest lattice Shapley - properties}

    \begin{block}{Claims $21$ - $24$}
        For integer game $G_I \in \mathcal{G}_I^n$, the CLS value $\phi^\mathcal{W}(G_I)$:
        \begin{itemize}
            \item depends on the choice of the norm.
        \end{itemize}
    \end{block}

    \begin{block}{$G_I^{\star}=(N,w_I)$}
        \begin{itemize}
            \item $N=\{1,2,3\}$,
            \item $w_I(\{1\})=w_I(\{1,2\})=w_I(\{1,3\})=1$,
            \item $w_I(\{2\})=w_I(\{3\})=w_I(\{2,3\})=0$,
            \item $w_I(N)=2$
        \end{itemize}
    \end{block}
\end{frame}

%------------------------------------------------

%------------------------------------------------

\begin{frame}{Closest lattice Shapley - properties}

    \begin{block}{Claims $21$ - $24$}
        For integer game $G_I \in \mathcal{G}_I^n$, the CLS value $\phi^\mathcal{W}(G_I)$:
        \begin{itemize}
            \item depends on the choice of the norm.
        \end{itemize}
    \end{block}

    \begin{itemize}
        \item $\phi(w_i) = (\frac{4}{3},\frac{1}{3},\frac{1}{3})$,
        \item $||x||_A=\sqrt{x^T A x}\implies \phi^\mathcal{W}_A(w_I) = (2,0,0)$
        \begin{itemize}
            \item $A = \begin{pmatrix}
            2 & 0 & 0 \\
            0 & 3 & 0 \\
            0 & 0 & 4
        \end{pmatrix}$
        \item Distance (in $A$-norm): $1.29099$
        \end{itemize}
        \item $\ell_1$-norm$\implies \phi^\mathcal{W}_{\ell_1}(w_I) = (1,1,0)$
        \begin{itemize}
            \item Distance (in $A$-norm): $1.41421$
        \end{itemize}
    \end{itemize}
\end{frame}

%------------------------------------------------

%------------------------------------------------

\begin{frame}{Integer nucleolus}
    %\pause

    \begin{block}{Integer nucleolus}
        For an integer cooperative game $G_I \in \mathcal{G}_I^n$, the \textbf{integer nucleolus} $\eta_\Z(G_I)$ is defined as
        \begin{displaymath}
            \eta_{\Z}(G_I) = \left\{ x \in \mathcal{I}_\Z(G_I) \mid \forall y \in \mathcal{I}_\Z(G_I): \Theta(x) \preceq_{lex} \Theta(y)\right\}.
        \end{displaymath}
    \end{block}

    %\pause

    \begin{itemize}
        \item When does it exist?
    \end{itemize}

    %\pause

    \begin{block}{Theorem $25$ - $26$}
        For an integer cooperative game $G_I \in \mathcal{G}_I^n$, it holds

        \begin{displaymath}
            \mathcal{I}_{\Z}(G_I) \neq \emptyset \implies \eta_\Z(G_I) \neq \emptyset,
        \end{displaymath}

        additionally it holds that

        \begin{displaymath}
            \eta_\Z(G_I) \neq \emptyset \iff \eta(G_I) \neq \emptyset.
        \end{displaymath}
    \end{block}
\end{frame}

%------------------------------------------------

%------------------------------------------------

\begin{frame}{Integer nucleolus - properties}
    %\pause

    \textit{We lose the uniqueness...}

    %\pause
    \begin{block}{Theorem $27$}
        For an integer cooperative game $G_I \in \mathcal{G}_I^n$ the nucleolus is not necessarily a single point solution concept.
    \end{block}
\end{frame}

%------------------------------------------------

%------------------------------------------------

\begin{frame}{Integer classes}
    %\pause
    \begin{itemize}
        \item Classes - games whose characteristic function satisfies certain properties
        %\pause
        \item Introduced new integer based classes, for given $c$:
        \begin{itemize}
            \item $c$-tight - $\forall S \subseteq N: 0 \leq v(S) \leq c \land v(N) = c$
            \item $c$-bounded - $\forall S \subseteq N: 0 \leq v(S) \leq c$
        \end{itemize}
        %\pause
        \item Experiments:
        \begin{itemize}
            \item Counted games from standard classes (e.g. convex games) that are also $c$-bounded/tight
            \item Derived exact formulae for positive and $k$-games
        \end{itemize}
    \end{itemize}
\end{frame}

%------------------------------------------------

%----------------------------------------------------------------------------------------
%	 SECTION 4
%----------------------------------------------------------------------------------------

\section{Library}

%------------------------------------------------

\begin{frame}{Python library}
    %\pause
    \begin{itemize}
        \item \textbf{Shapleypy} - Python library for standard cooperative game theory
        %\pause
        \item Available via PyPI or source code on GitHub
        %\pause
        \item Currently offers:
        \begin{itemize}
            \item Solution concepts - Shapley and Banzhaf value, core
            \item Classes - recognition of standard classes of games, generation of some classes
            \item I/O - possibility to load and save games
        \end{itemize}
        %\pause
        \item Future plans:
        \begin{itemize}
            \item More solution concepts - nucleolus, kernel, etc.
            \item More generators - might not be complete uniformly random
            \item Better documentation
            \item Include integer results
        \end{itemize}
    \end{itemize}
    
\end{frame}

%------------------------------------------------

\section{Thank you for your attention}

\end{document}
